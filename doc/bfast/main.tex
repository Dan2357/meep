\documentclass{article}
\usepackage{graphicx} % Required for inserting images

\title{Fixed angle broadband simulations in MEEP}
\author{Daniel Lloyd-Jones }
\date{28th July 2023}
\usepackage{amsmath}
\begin{document}

\maketitle

\section{Introduction}
Currently in MEEP, Bloch Periodic boundary conditions are implemented, which fix the wave vector of an incident wave \cite{MEEP}. As a result, the angle of an oblique incident wave becomes frequency dependent. Following the procedure detailed by B. Liang et al \cite{BFAST}, all fields can be redefined so that the boundary conditions become periodic and the angle of the incident wave can be fixed over a broad frequency spectrum. This requires the addition of a new field. It is assumed that the reader is already familiar with the UPML formulation in MEEP \cite{UPML}, from which the equations will be modified.

\section{Boundary conditions}
The fields from section 3 of \emph{Notes on the UPML implementation in MEEP} \cite{UPML} are first redefined as:
\begin{equation} \label{redef}
\text{field}'(x,y,z) = \text{field}(x,y,z)e^{-i(k_{x}x+k_{y}y)},
\end{equation}
where $k_{x}$ and $k_{y}$ are the wave vector components in the x and y directions. This is for a structure which is periodic in these directions. Taking the electric field $E$ as an example, the new boundary condition can be expressed as
\begin{equation}
E'(x+a,y+b,z) = E(x+a,y+b,z)e^{-i(k_{x}(x+a)+k_{y}(y+b))}
\end{equation}
where a is the length of the unit cell in the x direction and b in the y direction. Substituting in the original Bloch periodic boundary conditions gives
\begin{equation}
E'(x+a,y+b,z) = E(x,y,z)e^{i(k_{x}a+k_{y}b)}e^{-i(k_{x}(x+a)+k_{y}(y+b))}.
\end{equation}
Cancelling the $a$ and $b$ terms gives
\begin{equation}
E'(x+a,y+b,z) =E(x,y,z)e^{-i(k_{x}x+k_{y}y)}=E'(x,y,z),
\end{equation}
and so the boundary conditions are now periodic.

\section{Formulation}
Equation (5) from section 3 of \emph{Notes on the UPML implementation in MEEP} \cite{UPML} is
\begin{equation} \label{K}
\vec{K} = \nabla \times \vec{H}=-i\omega(1+\frac{i\sigma_{D}}{\omega})\vec{C},
\end{equation}
where $\vec{H}$ is the magnetic field, $\sigma_{D}$ the conductivity and $\vec{C}$ an auxiliary field. When the magnetic field is redefined, the curl of a product must be carried out:
\begin{equation}
\nabla\times \vec{H'} = \nabla\times (\vec{H} e^{-i(k_{x}x+k_{y}y)})
\end{equation}
so,
\begin{equation}
\nabla\times \vec{H'} = e^{-i(k_{x}x+k_{y}y)} \nabla\times \vec{H} + \begin{pmatrix} -ik_{x} \\-ik_{y}\\0 \end{pmatrix} \ \times \vec{H'}
\end{equation}
where the complex exponential in the second term has been absorbed by $\vec{H'}$.
Substituting in equation (\ref{K}) gives
\begin{equation} \label{h_prime}
\nabla\times \vec{H'} = \vec{K'} = -i\omega (1+\frac{i\sigma_{D}}{\omega}) \vec{C'} + \begin{pmatrix} -ik_{x} \\-ik_{y}\\0 \end{pmatrix} \ \times \vec{H'}.
\end{equation}
From here on in, the prime notation can be dropped since this applies to all fields. By introducing a new field $\vec{F}$, equation (\ref{h_prime}) can be written as
\begin{equation} \label{new_k}
\vec{K} = -i\omega(1+\frac{i\sigma_{D}}{\omega})\vec{C} - i\omega\vec{F}.
\end{equation}
This new field satisfies the equation:
\begin{equation} \label{F}
\vec{F} = \vec{\bar{k}}\times\vec{H},
\end{equation}
where
\begin{equation}
\vec{\bar{k}} =\frac{1}{\omega}\begin{pmatrix} k_{x} \\ k_{y}\\0 \end{pmatrix} \ =\begin{pmatrix} \sin{\theta}\cos{\phi} \\ \sin{\theta}\sin{\phi}\\0 \end{pmatrix} \
\end{equation}
and so $\vec{\bar{k}}$ is the wave vector with its frequency dependence removed. $\theta$ and $\phi$ are the propagating direction angles and c, the speed of light is taken to be 1. Therefore by defining $\vec{F}$, the angle of the incident wave is fixed.
Equation (\ref{F}) can be discretized as:
\begin{equation}
\vec{F}^{n+1}=2\bar{\vec{k}}\times\vec{H}^{n+0.5} -\vec{F}^{n} .
\end{equation}
Transforming equation (\ref{K}) to the time domain gives:
\begin{equation}
\vec{K} = \frac{\partial \vec{C}}{\partial t}+\sigma_{D}\vec{C}+\frac{\partial \vec{F}}{\partial t} .
\end{equation}
This can be discretized as:
\begin{equation} \label{disc_k}
\vec{K}^{n+0.5}=\frac{\vec{C}^{n+1}-\vec{C}^n}{\Delta t}+\sigma_{D}\frac{\vec{C}^{n+1}+\vec{C}^n}{2} + \frac{\vec{F}^{n+1}-\vec{F}^{n}}{\Delta t}
\end{equation}
and then solved to update the value of $\vec{C}$ using:
\begin{equation}
\vec{C}^{n+1}=(1+\frac{\sigma_{D} \Delta t}{2})^{-1} [(1-\frac{\sigma_D \Delta t}{2}) \vec{C}^n+\Delta t\vec{K}^{n+0.5}+\vec{F}^{n}-\vec{F}^{n+1}] .
\end{equation}
All other equations are unaffected by these changes.

A new field must be introduced because $\vec{H}$ is defined at $n+\frac{1}{2}$ timesteps whereas $\vec{C}$ is defined at $n$ timesteps, where $n$ is an integer. As a result, if the derivative in $\vec{F}$ in equation (\ref{disc_k}) was replaced with
\begin{equation}
\vec{\bar{k}}\times(\frac{\vec{H}^{n+0.5}-\vec{H}^{n-0.5}}{\Delta t}),
\end{equation}
only first order accuracy would be achieved, since this is a backward difference scheme. To achieve second order accuracy would require $\vec{H}^{n+1.5}$ to be known.

\section{Stability}
As the incident angle increases, the maximum possible $\Delta t$ value decreases, following the formula:
\begin{equation}
\frac{c\Delta t}{\Delta x} \leq \frac{(1-sin(\theta))}{\sqrt{D}}
\end{equation}
where D is the number of dimensions \cite{BFAST}.

\begin{thebibliography}{9}
\bibitem{MEEP}
Taflove A., Oskooi A., Johnson S.. \emph{Advances in FDTD Computational Electrodynamics: Photonics and Nanotechnology}.  Artech House, Inc.; 2013

\bibitem{BFAST}
Liang B., Bai M., Ma H., Ou N., Miao J.. Wideband Analysis of Periodic Structures at Oblique Incidence by Material Independent FDTD Algorithm. \emph{IEEE Transactions on Antennas and Propagation}, vol. 62, no. 1, pp. 354-360, Jan. 2014, doi: 10.1109/TAP.2013.2287896.

\bibitem{UPML}
Johnson S. \emph{Notes on the UPML implementation in Meep}. Massachusetts Institute of Technology. Posted August 17, 2009; updated March 10, 2010. http://ab-initio.mit.edu/meep/pml-meep.pdf

\end{thebibliography}
\end{document}
